%!TEX ROOT=./fl_main.tex

\section{Introduction}
\label{sec:intro}

% DL

%Repair paper

Fault localisation (FL) in DNNs is a rapidly evolving area of DL testing~\cite{deeplocalize, wardat2022deepdiagnosis, deepfd,
nikanjam2021automatic, schoop2021umlaut}. The decision logic of traditional software systems is encoded in their source code. Correspondingly, fault localisation for such systems consists of identifying the parts of code that are most likely responsible for the encountered misbehaviours. Unlike traditional software systems, the decision logic of DL systems depends on many components such as the model structure, selected hyper-parameters, training dataset and the framework used to perform the training process. Moreover, DL systems are stochastic in nature, as a retraining with the exactly same parameters might lead to a slightly different final model and performance. These distinctive characteristics make the mapping of a misbehaviour (e.g., poor classification accuracy) to a specific fault type a highly challenging task.

Existing works~\cite{deeplocalize,autotrainer,wardat2022deepdiagnosis,bakerdetect,nikanjam2021automatic} that focus on the problem of fault localisation for DL systems rely on  patterns of inefficient model structure design, as well as a set of predefined rules about the values of internal indicators measured during the  DL training process. This makes the effectiveness of these approaches highly dependent on the identified set of rules and on the threshold values selected to discriminate the values of the internal indicators  of a fault.

To understand whether these tools effectively generalise to a diverse set of fault types and DL systems, and thus, are effective for the real-world usage, we performed an empirical evaluation on a benchmark of carefully selected subjects. In this benchmark we combined faults obtained by the artificial injection of defects into otherwise well-performing DL models and a set of reproduced real DL faults. This way we ensure that our evaluation involves model of different structures and complexity that solve problems from different domains. 

Our results show that existing DNN FL techniques produce stable results in a relatively small amount of time. However, accuracy of fault localisation techniques should still be the focus of future research.
