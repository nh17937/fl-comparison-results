\documentclass[conference]{IEEEtran}
\IEEEoverridecommandlockouts
% The preceding line is only needed to identify funding in the first footnote. If that is unneeded, please comment it out.
\usepackage{cite}
\usepackage{amsmath,amssymb,amsfonts}
%\usepackage{algorithm}
\usepackage{graphicx}
\usepackage{textcomp}
\usepackage{xcolor}
\usepackage{booktabs}
\usepackage{listings}
\usepackage{multirow}
\usepackage{caption}
\usepackage{subcaption}
\usepackage{framed}

\usepackage{soul}
\usepackage[utf8]{inputenc}
\usepackage[T1]{fontenc}
\usepackage{microtype}
\usepackage{rotating}
\usepackage{paralist}
\usepackage{tabularx}
\usepackage{multicol}
\usepackage{pbox}
\usepackage{enumitem}	
\usepackage{colortbl}
\usepackage{pifont}
\usepackage{xspace}
\usepackage{url}
\usepackage{tikz}
\usepackage{fontawesome}
\usepackage{lscape}
\usepackage{color}
\usepackage{anyfontsize}
\usepackage{comment}
\usepackage{soul}
\usepackage{gensymb}
\usepackage{multibib}
\usepackage{tcolorbox}
\usepackage{balance}
\usepackage{footmisc}
\usepackage{tcolorbox}

\lstset{columns=fullflexible}

\usepackage[ruled,linesnumbered]{algorithm2e}
\def\BibTeX{{\rm B\kern-.05em{\sc i\kern-.025em b}\kern-.08em
    T\kern-.1667em\lower.7ex\hbox{E}\kern-.125emX}}

\usepackage{xspace}

\definecolor{box-white}{cmyk}{0, 0, 0, 0, 0}

%\newcommand{\tool}{\textsc{DeepRepair}\xspace} 
%\newcommand{\dc}{\textsc{DeepCrime}\xspace} 
%\newcommand{\AutoT}{\textsc{AutoTrainer}\@\xspace}
%\newcommand{\dfd}{\textsc{DeepFD}\xspace}
\newcommand{\fixme}[1]{\textcolor{red}{\textbf{FIXME:}#1}}

\newboolean{showcomments}
\setboolean{showcomments}{true}         
%\setboolean{showcomments}{false} 
\ifthenelse{\boolean{showcomments}}
  {\newcommand{\nb}[2]{
  \fbox{\bfseries\sffamily\scriptsize#1}
     {\sf\small$\blacktriangleright$\textit{\textcolor{red}{#2}}$\blacktriangleleft$}
   }
  }
  {\newcommand{\nb}[2]{}
   \newcommand{\cvsversion}{}
  }

%\newcommand\fixme[1]{\nb{FIXME:}{#1}}
% \newcommand\new[1]{{\color{blue}#1}} 
\newcommand\new[1]{{\color{black}#1}} 
\newcommand\shin[1]{\nb{Shin}{#1}} 
\newcommand\han[1]{\nb{Jinhan}{#1}} 
\newcommand\paolo[1]{\nb{Paolo}{#1}} 
\newcommand\gunel[1]{\nb{Gunel}{#1}}
\newcommand\nargiz[1]{\nb{Nargiz}{#1}}

\newcommand{\DC}{\textsc{DeepCrime}\@\xspace}
\newcommand\DM{\textsc{DeepMetis}\xspace}
\newcommand{\DMPP}{\textsc{DeepMutation++}\@\xspace}
\newcommand{\AutoT}{\textsc{AutoTrainer}\@\xspace}
\newcommand{\dfd}{\textsc{DeepFD}\xspace}
\newcommand{\DL}{\textsc{DeepLocalize}\xspace}
\newcommand{\DD}{\textsc{DeepDiagnosis}\xspace}
\newcommand{\UM}{\textsc{UMLAUT}\xspace}
\newcommand{\NL}{\textsc{Neuralint}\xspace}
\newcommand{\DLF}{\textsc{DLFuzz}\xspace}
\newcommand{\DJS}{\textsc{DeepJanus}\xspace}
\newcommand{\HB}{\textsc{HEBO}\xspace}
\newcommand{\BHB}{\textsc{BOHB}\xspace}

\newcommand{\etal}{\textit{et al.}\xspace}


\newcommand{\COMMENT}[1]{}

\definecolor{codegreen}{RGB}{0, 160, 0}
\definecolor{codered}{RGB}{160, 0, 0}

\newcommand*{\codegreen}{\lstinline[keywordstyle=\color{codegreen}, basicstyle=\color{codegreen}]}
\newcommand*{\codered}{\lstinline[keywordstyle=\color{codered}, basicstyle=\color{codered}]}


\begin{document}

\title{An Empirical Study of Fault Localisation Techniques for Deep Learning}

\makeatletter
\newcommand{\linebreakand}{%
  \end{@IEEEauthorhalign}
  \hfill\mbox{}\par
  \mbox{}\hfill\begin{@IEEEauthorhalign}
}
\makeatother

 \author{\author{\IEEEauthorblockN{Anonymous Author(s)}}}
%\author{\IEEEauthorblockN{Jinhan Kim}
%\IEEEauthorblockA{\textit{School of Computing}\\
%\textit{KAIST}\\
%Daejeon, Republic of Korea\\
%jinhankim@kaist.ac.kr}
%\and
%\IEEEauthorblockN{Nargiz Humbatova}
%\IEEEauthorblockA{\textit{Software Institute}\\
%\textit{Università della Svizzera italiana (USI)}\\
%Lugano, Switzerland \\
%nargiz.humbatova@usi.ch}
%\and
%\IEEEauthorblockN{Gunel Jahangirova}
%\IEEEauthorblockA{\textit{Department of Informatics}\\
%\textit{King's College London}\\
%London, UK\\
%gunel.jahangirova@kcl.ac.uk}
%\linebreakand
%\IEEEauthorblockN{Paolo Tonella}
%\IEEEauthorblockA{\textit{Software Institute}\\
%\textit{Università della Svizzera italiana (USI)}\\
%Lugano, Switzerland \\
%paolo.tonella@usi.ch}
%\and
%\IEEEauthorblockN{Shin Yoo}
%\IEEEauthorblockA{\textit{School of Computing}\\
%\textit{KAIST}\\
%Daejeon, Republic of Korea\\
%shin.yoo@kaist.ac.kr}
%}

\maketitle

\begin{abstract}
Dummy abstract

\end{abstract}

\begin{IEEEkeywords}
deep learning, real faults, fault localisation
\end{IEEEkeywords}

%!TEX ROOT=./fl_main.tex

\section{Introduction}
\label{sec:intro}


%!TEX ROOT=./fl_main.tex

\section{Automated Fault Localisation for FL}
\label{sec:repair}

%%!TEX ROOT=./fl_main.tex

\section{Benchmark}
\label{sec:benchmark}

%!TEX ROOT=./fl_main.tex

\section{Empirical study} \label{sec:empirical_study_fl}

\subsection{Research Questions}

The \textit{aim} of this empirical study is to compare existing DL fault localisation approaches and to explore their generalisability to different subjects represented by our benchmark of artificial and real faults. To cover these objectives, we define the following research questions:

\begin{itemize}
    \item \textbf{RQ1. Effectiveness}: \textit{Can existing FL approaches identify and locate defects in faulty DL models? Which FL tool produces the most accurate and actionable result?}
    \item \textbf{RQ2. Stability}: \textit{Is the outcome of fault identification analysis stable across several runs?}
    \item \textbf{RQ3. Efficiency}: \textit{How costly are FL tools when compared to each other?}
\end{itemize}


\subsection{Benchmark}

\nargiz{describe, cite repair paper}

\subsection{Experimental Settings \& Evaluation Metrics} \label{sec:exp_fl}
For the comparison we use publicly available versions of all considered tools~\cite{deepfd_replication, umlaut_replication, neuralint_replication, deepdiagnosis_replication}. However, we had to limit the artificial faults to those obtained using CIFAR10, MNIST, and Reuters as \DD is not applicable to other subjects.

The authors of \dfd adopted the notion of statistical mutation killing~\cite{JahangirovaICST20} in their tool. They run each of the models used to train the classifier as well as the model under test 20 times to collect the run-time features. For the fault localisation using \dfd, we adopt an ensemble of already trained classifiers provided in the tool's replication package. Similar to the authors, for each faulty model in our benchmark, we collect the run-time behavioural features from 20 retrainings of the model. \NL is based on static checks that do not require any training and thus, are not prone to randomness. We run each of the remaining tools 20 times to account for the randomness in training process and report the most frequently observed result.

To calculate the similarity between the ground truth provided for each fault in our benchmark and the fault localisation results, we adopt the Asymmetric Jaccard metric that was used in the empirical study of repair tools~(see Section~\ref{sec:eval_metric_rep}). In this case, the metric measures the percentage of the fault types in the list of localised faults  ($OP_{loc}$) that are also found in the ground truth ($OP_{gt}$):
\begin{equation}
    AJ = \frac{| OP_{loc} \cap OP_{gt} |}{| OP_{gt} |}
\end{equation}


%!TEX ROOT=./fl_main.tex

\section{Results}
\label{sec:results}
%This section presents the results of the empirical study and answers each research question.

%%!TEX ROOT=./fl_main.tex

\section{Discussion}
\label{sec:discussion}

%!TEX ROOT=./fl_main.tex

\section{Threats to Validity}
\label{sec:threats}

%!TEX ROOT=./fl_main.tex

\section{Related Work}
\label{sec:related_work}

%!TEX ROOT=./fl_main.tex

\section{Conclusion}
\label{sec:conclusion}

We evaluated 4 state-of-the-art techniques in DL fault localisation on a  meticulously tailored set of real and artificial faulty models to asses the advances in the area. Our findings show that all of the evaluated approaches are able to locate a certain percentage of faults. However, all are quite far from the best possible results. \dfd exhibited the highest effectiveness, followed by \NL and \UM. \DD exhibited relatively poor performance. On the positive side, all proposed techniques are stable across multiple runs and do not require substantial resources. According to our findings, future work in the area of DNN fault identification and localisation should focus on improving the accuracy of the proposed techniques. This suggests that it is highly unlikely that existing approaches for DNN tuning and repair might significantly benefit from the combination with the evaluated localisation tools.
%!TEX ROOT=./fl_main.tex

\section*{Acknowledgement}
Jinhan Kim and Shin Yoo have been supported by the Engineering Research Center
Program through the National Research Foundation of Korea (NRF) funded by the
Korean Government (MSIT) (NRF-2018R1A5A1059921), NRF Grant (NRF-2020R1A2C1013629),
Institute for Information \& communications Technology Promotion grant funded by
the Korean government (MSIT) (No.2021-0-01001), and Samsung Electronics (Grant
No. IO201210-07969-01). This work was partially supported by the H2020 project
PRECRIME, funded under the ERC Advanced Grant 2017 Program (ERC Grant Agreement
n. 787703).

%\bibliographystyle{IEEEtran}
%\bibliography{biblio,newref}
\bibliographystyle{IEEEtran}
\balance
%\bibliography{IEEEabrv,biblio,newref}
\bibliography{IEEEabrv,biblio2}

\vspace{12pt}
\end{document}
